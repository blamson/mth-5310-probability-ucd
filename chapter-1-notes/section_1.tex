\section*{Set Theory}

\begin{definition}[Sample Space]
	The set $S$, of all possible outcomes of a particular experiment is called the \textbf{sample space} for the experiment.
	\label{Sample Space}
\end{definition}

Notation: Sample Space $= S = \Omega$

Example: Single coin flip. $S = \{H,T\}$

\begin{definition}[Event]
	An \textbf{event} is any collection of possible outcomes of an experiment. That is, any possible subset of $S$ (including $S$ itself).
\end{definition}

Example: An experiment where a coin is flipped two times consequtively.

$S = \left\{ HH,HT,TH,TT \right\}$

Let $A = \text{First flip is heads} = \left\{ HH, HT \right\}$

\subsection*{Elementary Set Operations}

\begin{definition}[Union]
	The \textbf{union} of $A$ and $B$ are the set of elements that belong to either $A$ OR $B$.
\end{definition}

Notation: $A \cup B$

$A \cup B = \left\{ x:x\in A or x \in B \right\}$

\begin{definition}[Intersection]
	The \textbf{intersection} of $A$ and $B$ are the set of elements that belong to both $A$ AND $B$.
\end{definition}

Notation: $A \cap B$

$A \cap B = \left\{ x: x \in A and x \in B \right\}$

\begin{definition}[Complement]
	The \textbf{complement} of $A$ is the set of all elements not in $A$.
\end{definition}

Notation: $A^c$

$A^c = \left\{ x: x \notin A \right\}$


\begin{theorem}
	1.1.4: Useful Set Properties

	For any events $A, B, C$, defined on a sample space $S$.

	Commutativity:

	\begin{align*}
		A \cup B &= B \cup A\\
		A \cap B &= B \cap A
	\end{align*}

	Associativity:

	\begin{align*}
		A \cup (B \cup C) &= (A \cup B) \cup C \\
		A \cap (B \cap C) &= (A \cap B) \cap C
	\end{align*}

	Distributive Laws:

	\begin{align*}
		A \cap (B \cup C) &= (A \cap B) \cup (A \cap C) \\
		A \cup (B \cap C) &= (A \cup B) \cap (A \cup C)
	\end{align*}

	DeMorgan's Laws:

	\begin{align*}
		A \cup (B \cup C) &= (A \cup B) \cup C \\
		A \cap (B \cap C) &= (A \cap B) \cap C
	\end{align*}

\end{theorem}

\begin{definition}[Disjoint, Pairwise Disjoint]
	Two events $A$ and $B$ are disjoint or mutually exclusive if $A \cap B = \emptyset$. The events $A_1, A_2, \cdots$ are pairwise disjoint if $A_i \cap A_j = \emptyset \; \forall \; i \neq j$
	
\end{definition}

\begin{definition}[Partition]
	If $A_1, A_2, \cdots$ are pairwise disjoint and $\cup_{i=1}^{\infty} A_i = S$ then the collection $A_1, A_2, \cdots$ forms a partition of $S$.
\end{definition}
