\subsection*{2.6}

\textbf{In each of the following find the pdf of Y and show that the pdf integrates to 1}.

I'm too lazy to type it out but we'll be using theorem 2.1.8 for this problem.

\subsubsection*{Part A}

$f_X(x) = \frac{1}{2}e^{-|x|}$, $-\infty < x < \infty$

\noindent $Y = |X|^3$

Big first step is to split up our domain. The absolute value requires we that examine the cases where $X<0$ and where $x>0$. You can proceed without doing this and still get something that resembles a pdf, but it won't integrate to 1. Instead it will go to 0.5 as that function will only capture half the possible values of $x$. Ask me how I know!

\begin{align*}
	A_0 &= \left\{ 0 \right\} \\
	A_1 &= (-\infty, 0) & g_1(x) &= |x|^3 & g_1^{-1}(y) &= -y^{-1/3} & \frac{d}{dy} g_1^{-1} &= -\frac{1}{3}y^{-2/3} \\
	A_2 &= (0, \infty) & g_2(x) &= |x|^3 & g_2^{-1}(y) &= y^{-1/3} & \frac{d}{dy} g_2^{-1} &= \frac{1}{3}y^{-2/3} \\
\end{align*}

Also worth noting that, since $x$ is wrapped in an absolute value, $y$ will always be greater than 0. As such:

\noindent $\mathcal{Y} = (0, \infty)$.

Now we have all of our pieces, we can simply partition out the formula provided by theorem 2.1.5 and work through it!

\begin{align*}
	\sum_{i=1}^k f_X(g_i^{-1}(y)) \left| \frac{d}{dy}g_i^{-1}(y) \right| 
	&= \left( \frac{1}{2} e^{-|-y^{1/3}|} \left| -\frac{1}{3}y^{-2/3} \right| \right) 
	\cdot \left( \frac{1}{2} e^{-|y^{1/3}|} \left| \frac{1}{3}y^{-2/3} \right| \right) 
	\\
	&= \frac{1}{6} e^{-y^{1/3}}y^{-2/3} + \frac{1}{6} e^{-y^{1/3}}y^{-2/3} \\
	&= \frac{1}{3} e^{-y^{1/3}}y^{-2/3}
\end{align*}

\[
	f_Y(y) = \begin{cases}
		 \frac{1}{3} e^{-y^{1/3}}y^{-2/3} & 0 < y < \infty \\
		 0 & \text{o/w}
	\end{cases}
\]

Lastly, we verify that this pdf does in fact evaluate to 1.

\[\int_{0}^{\infty} \frac{1}{3} e^{-y^{1/3}}y^{-2/3} dy\]

For this integral we'll do some u-substitution. 

$u=y^{1/3}$

$u^3 = y$

$3u^2 = dy$

\begin{align*}
	\int_{0}^{\infty} \frac{1}{3} e^{-y^{1/3}}y^{-2/3} dy &= \frac{1}{3} \int_{0}^{\infty} e^{-u} u^{-2} 3u^{2} du \\
	&= \int_{0}^{\infty} e^{-u} du \\
	&= -e^{-u} \rvert^{\infty}_0 \\
	&= -e^{-y^{1/3}} \rvert^{\infty}_0 \\	
	&= \lim_{y \to \infty} -e^{-y^{1/3}} - (-e^{-0^{1/3}}) \\
	&= -0 + 1 \\
	&= 1
\end{align*}

\pagebreak

\subsubsection*{Part B}

$f_X(x) = \frac{3}{8}(x+1)^2$, $-1 < x < 1$

\noindent $Y = 1 - X^2$

First let's solve for X.

\begin{align*}
	y &= 1 - x^2 \\
	y + x^2 &= 1 \\
	x^2 &= 1 - y \\
	x &= \pm \sqrt{1-y}
\end{align*}

Now we collect all the information we'll need.

\begin{align*}
	A_0 &= \left\{ 0 \right\} \\
	A_1 &= (-1, 0) & g_1^{-1}(y) &= -\sqrt{1-y} & \frac{d}{dy}g_1^{-1}(y) &= \frac{1}{2}(1-y)^{-1/2}\\
	A_2 &= (0, 1)  & g_2^{-1}(y) &= \sqrt{1-y} & \frac{d}{dy}g_2^{-1}(y) &= -\frac{1}{2}(1-y)^{-1/2}
\end{align*}

Lastly, for $\mathcal{Y}$, we solve for that by examining $Y = 1 - X^{2}$. Take my word that the minimum of this, given the possible values of x, is 0 and the max is 1.

$\mathcal{Y} = (0,1)$

Now for the meat of the problem. First we create our PDF, then we verify that it evaluates to 1.

\begin{align*}
	\sum_{i=1}^k f_X(g_i^{-1}(y)) \left| \frac{d}{dy}g_i^{-1}(y) \right| 
	&= \frac{3}{8} (-\sqrt{1-y} + 1)^2 \cdot \left| \frac{1}{2\sqrt{1-y}} \right| + \frac{3}{8} (\sqrt{1-y} + 1)^2 \cdot \left| \frac{1}{-2\sqrt{1-y}} \right| \\
	&= \frac{3}{8} \cdot \frac{1}{2\sqrt{1-y}} \cdot ( (-\sqrt{1-y} + 1)^2 + (\sqrt{1-y} + 1)^2 ) \\
	&= \frac{3}{8} \cdot \frac{1}{2\sqrt{1-y}} \cdot (2 - y - 2\sqrt{1-y} + 2 - y + 2\sqrt{1-y}) \\
	&= \frac{3}{8} \cdot \frac{1}{2\sqrt{1-y}} \cdot (4 - 2y) \\
	&= \frac{3}{8} \cdot \frac{1}{2} \cdot \frac{1}{\sqrt{1-y}} \cdot 2(2 - y) \\
	&= \frac{3}{8} \cdot (1-y)^{-1/2} \cdot (2-y) 
\end{align*}

Now we can build our PDF.

\[
	f_Y(y) = \begin{cases}
		 \frac{3}{8} \cdot (1-y)^{-1/2} \cdot (2-y) & 0 < y < \infty \\
		 0 & \text{o/w}
	\end{cases}
\]

A quick sanity check with a calculator indicates that this does evaluate to 1. So that's good! Now we can confidently evaluate the integral.

\[
	\frac{3}{8} \int_0^1 (1-y)^{-1/2} (2-y) dy
\]

To evaluate this we'll need to do a u-substitution. So let's get that out of the way. I will not be showing all of my work here, I want to be able to sleep. 

\begin{align*}
	u &= (1-y)^{1/2} \\
	y &= 1 - u^2 \\
	du &= -\frac{1}{2} (1-y)^{-1/2} dy \\
	dy &= -2du(1-y)^{1/2}
\end{align*}

Alright, let's dive in.

\begin{align*}
	\frac{3}{8} \int_0^1 (1-y)^{-1/2} (2-y) dy &= \frac{3}{8} \int_0^1 \frac{1}{(1-y)^{1/2}} (2-y) dy \\
	&= \frac{3}{8} \int_0^1 \frac{1}{(1-y)^{1/2}} (2-y) \cdot (-2du(1-y)^{1/2}) \\
	&= -\frac{6}{8} \int_0^1 (2-y) du \\
	&= -\frac{6}{8} \int_0^1 (2-(1 - u^2)) du \\
	&= -\frac{6}{8} \int_0^1 1 + u^2 du \\
	&= -\frac{6}{8} u + \frac{u^3}{3} \bigg\rvert_{y=0}^{y=1} \\
	&= -\frac{6}{8} \left( \sqrt{1-y} + \frac{(1-y)^{3/2}}{3} \right)\bigg\rvert_{y=0}^{y=1} \\
	&= -\frac{6}{8} \left(0 + 0 - \left(1 + \frac{1}{3}\right)\right) \\
	&= -\frac{6}{8} \left(- \frac{4}{3}\right) \\
	&= 1
\end{align*}
