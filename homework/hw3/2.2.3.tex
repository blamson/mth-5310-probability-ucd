\subsection*{2.3}

Suppose $X$ has a geometric pmf, $f_X(x) = \frac{1}{3}\left( \frac{2}{3} \right)^x$, $x=0,1,2,3,\cdots$. Determine the probability distribution of $Y = X/(X+1)$. Note here that both $X$ and $Y$ are discrete random variables. To specify the probability distribution of $Y$, specify its pmf.

Our key here is this: 

\[
	f_Y(y) = P(Y=y) = \sum_{x \in g^{-1}(y)}P(X=x) = \sum_{x \in g^{-1}(y)} f_X(x), \; \text{for} \; y \in \mathcal{Y}
\]

Essentially what this means is, for a given value of y, we find all the x's that map to that value and sum all of those probabilities up.

First let's find the domain of Y.

\[
	\mathcal{Y} = \left\{ y: y=g(x), x \in \mathcal{X} \right\}
\]

Because $y = g(x) = \frac{x}{x+1}$ we can simply get our domain by plugging in our possible values of x.

\[\mathcal{Y} = \left\{ 0, \frac{1}{2}, \frac{2}{3}, \frac{3}{4}, \cdots \right\}\]

Now we need $g^{-1}(y)$

\begin{align*}
	y &= \frac{x}{x+1} \\
	y &= x \cdot \frac{1}{x + 1} \\
	y(x+1) &= x \\
	yx + y &= x \\
	yx &= x-y \\
	yx - x &= -y \\
	x(y-1) &= -y \\
	x &= \frac{-y}{y-1} \\
	g^{-1}(y) &= \frac{-y}{y-y}
\end{align*}

Now we just plug in our map into $f_X(x)$. Because only one x maps to each y the sum will end up going away.

\begin{align*}
	f_Y(y) &= \sum_{x \in g^{-1}(y)}f_X(x) \\
	&= f_X\left( \frac{-y}{y-1} \right) \\
	f_Y(y) &= \frac{1}{3} \left( \frac{2}{3} \right)^{\frac{-y}{y-1}}, \; y \in \left\{ 0, \frac{1}{2}, \frac{2}{3}, \ldots \right\}
\end{align*}


