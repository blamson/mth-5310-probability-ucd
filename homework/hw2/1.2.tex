\subsection*{2}

An urn contains 5 red balls and 2 green balls. A single ball is randomly drawn from the urn. If the ball is green, then a red ball is added to the urn. If the ball is red, then a green ball is added to the urn. The original ball is not returned to the urn no matter if it is green or red. Now, a second ball is randomly drawn from the urn.

\subsubsection*{A}

In probability notation, write the P(second ball is red) as a function of the possible outcomes of the first draw. \\

\noindent \textbf{Solution:}

Let's set some groundwork here. First, let's create a basic pdf for our first draw.

\noindent \textbf{Notation:} Let $x=1$ if a red ball is pulled, $x=0$ otherwise.

\[
	f_X(x) = \begin{cases} 
		\frac{3}{8} & x=0 \\
		\frac{5}{8} & x=1
	\end{cases}
\]

This is of course just a Bernoulli random variable. Let's now look at how the urn changes for our second pull depending on what we pull. If $x=0$, our urn now has 6 red and 2 green. If $x=1$, we now have 4 red and 4 green. 

\noindent \textbf{Notation:} Let $A$ be the event that our second ball is red. 

To tackle this problem we can use teorem 1.2.11 which states that if P is a probability function, then $P(A) = \sum_{i=1}^{\infty} P(A \cap C_i)$ for any partition $C_1, C_2, \cdots$. We can clearly see that the first pull creates a parition of our sample space. 

So we can now write out our goal and work towards it.

\[P(A) = P(A \cap x=1) + P(A \cap x=0)\]

We don't know $P(A \cap x=1)$ directly, but we do know its building blocks. 

	\begin{align*}
		P(A|X=x) &= \frac{P(A \cap X=x)}{P(X=x)} \\
		P(A \cap X=x) &= P(A | X=x) \cdot P(X=x)
	\end{align*}

So now we can write this out as:

\[P(A) = P(A | x=1) \cdot P(x=1) + P(A | x=0) \cdot P(x=0)\]

\subsubsection*{B}

Calculate the probability that the second ball is red.

For our final building blocks, let's provide the conditional probabilities. These are easily pulled from the updated proportions on our second pull.

\begin{align*}
	P(A | X=0) &= \frac{6}{8} \\
	P(A | X=1) &= \frac{4}{8}
\end{align*}

Now we just plug it all in.

\begin{align*}
	P(A) &= P(A | x=1) \cdot P(x=1) + P(A | x=0) \cdot P(x=0) \\
	&= \frac{4}{8} \cdot \frac{5}{8} + \frac{6}{8} \cdot \frac{3}{8} \\
	&\approx 0.59375
\end{align*}

As a quick sanity check, doing this same process for the second pull being green nets a probabiity of $\approx 0.40625$ which does sum to 1 with our only other possible event. So we're done.
