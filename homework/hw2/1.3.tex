\subsection*{3}

A box contains two coins: (1) a regular fair coin that has P(Head)=P(Tail)=0.5 and (2) a fake two-headed coin where both sides are heads. That is, P(Head) = 1.

A single coin is chosen at random and then flipped twice. Describe the following events:

\begin{enumerate}
	\item A = First coin toss results in a Head
	\item B = Second coin toss results in a Head
	\item C = The regular coin was the chosen coin.
\end{enumerate}

\subsubsection*{A}

Explain in practical language, why A and B are not independent, but are conditionally independent given C.

A and B are not independent because the probability for B relies on the outcome of A. As an example, if you land a Tails on the first toss you automatically know you're using the fair coin. Thus, P(B) becomes a simple 0.5. If the first toss is a Heads the computation is more involved. The fact this changes at all depending on the first toss is why they aren't independent.

Now, if you know which coin you're using the tosses become independent, even with the unfair coin. A and B are conditionally independent on C because the outcome of the first toss gives you information on what coin you're using.

\subsubsection*{B}

Calculate $P(A|C), P(B|C), P(A \cap B | C), P(A), P(B), and P(A \cap B)$. Hint, calculate the probabilities in the order provided.

\begin{align*}
	P(A|C) &= 0.5 \\
	P(B|C) &= 0.5 \\
	P(A \cap B | C) &= 0.5^2= 0.25 \\
	P(A) &= P(A | C) \cdot P(C) + P(A | C^c) \cdot P(C^c) \\
	&= 0.5 \cdot 0.5 + 1 \cdot 0.5 \\
	&= 0.75 \\
	P(B) &= P(B | C) \cdot P(C) + P(B | C^c) \cdot P(C^c) \\
	&= 0.75 \\
	P(A \cap B) &= P(A \cap B | C) \cdot P(C) + P(A \cap B | C^c) \cdot P(C^c) \\
	&= 0.25 \cdot 0.5 + 1 \cdot 0.5 \\
	&= 0.625
\end{align*}



