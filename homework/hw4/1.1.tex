\subsection*{1}

Assume a random variable $X$ always equals a constant value $c$.

\subsubsection*{A}

Find the moment generating function for $X$

\begin{align*}
	M_X(t) &= E[e^{tx}] \\
	&= E[e^{tc}] \\
	&= e^{tc}
\end{align*}

The explanation here is that because $X=c \forall x$ the value inside the expectation becomes a constant as well. The expectation of a constant is just the constant. 

\subsubsection*{B}

Use the mgf to find the mean and variance of $X$

\begin{align*}
	E[X] &= \frac{d}{dt} M_X(t=0) & E[X^2] &= \frac{d^2}{dt^2} M_X(0) \\
	&= \frac{d}{dt} e^{tc} \bigg\rvert_{t=0} & &= \frac{d}{dt} ce^{tc} \bigg\rvert_{t=0} \\
	&= ce^{0c} & &= c^2 e^{0c} \\
	&= c & &= c^2
\end{align*}

\begin{align*}
	Var[X] &= E[X^2] - (E[X])^2 \\
	&= c^2 - c^2 \\
	&= 0
\end{align*}

\subsubsection*{C}

Explain why these values are logical.

The expected value of a constant is a constant. So that ones straightforward enough I believe. A constant is also, well, constant. It doesn't vary at all. Therefore a variance of 0 makes sense.
