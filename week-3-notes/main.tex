%        File: main.tex
%     Created: Tue Sep 10 12:00 PM 2024 M
% Last Change: Tue Sep 10 12:00 PM 2024 M
%
\documentclass[a4paper]{article}
% Head related stuff
\usepackage{fancyhdr}
\pagestyle{fancy}
\fancyhead[R]{MTH 5310 - Probability}  
\fancyhead[L]{Week 2 Notes}               
\fancyfoot[R]{Brady Lamson}
\fancyfoot[L]{CU Denver}

% Math packages
\usepackage{amsmath}
\usepackage{amsthm}

% Creates quick matrix of any type ----
% NOTE: p for paran, b for brackets, B for curly braces, v for pipes, V for double pipes 
\newcommand{\m}[2]{\begin{#1matrix}#2\end{#1matrix}}

% Creates small p-matrix ----
\newcommand{\psmall}[1]{
	\left(\begin{smallmatrix}
		#1
	\end{smallmatrix} \right)
}

% makes mathbb commands faster to type
\newcommand{\bb}{\mathbb}

\newtheorem*{definition}{Definition}
\newtheorem*{theorem}{Theorem}

\begin{document}
\begin{definition}[1.3.7: Independence]
	Two events $A \& B$ are statistically independent IF AND ONLY IF $P(A \cap B) = P(A)P(B)$
\end{definition}

\begin{definition}[1.3.12]
	A collection of events $A_1, \cdots, A_k$ are mutually independent if for any subcollection $A_{i1}, \cdots, A_{ik}$
	\[P\left( \bigcap_{j=1}^{k} A_{ij} \right) = \prod_{i=1}^{k} P(A_{ii})\]
\end{definition}

\begin{definition}[1.4.1: Random Variables]
	A random variable is a function from a sample space, $S$, into the real numbers. In other words, a random variable is your map. 
\end{definition}

Takeaways:

- Outcomes or events must be quantifiable. 

- A RV is a map. 

\subsection{Example:}

$S = \left\{ S_1, \cdots, S_n \right\}$ w/ associated $\sigma-$algebra. 

Let $X$ be a RV with range $X = \left\{ x_1, \cdots, x_m \right\}$

Since we have a sample space and a valid sigma algebra, this allows us to define a valid probability function, called $P$.

Then the probability on $X$ (or $P_X$) can be defined as we observe $X=x_i$ iff the outcome of the random experiment is an:

\begin{align}
	s_i \in S \; s.t. \; X(s_i) \\
	P_X(X=x_i) = P(\left\{ s_i \in S : X(s_i) = x_i \right\})
\end{align}

Notes got super hard to read here. Board writing is WILD.

\end{document}


